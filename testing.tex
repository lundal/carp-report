\todo{Støvneng verified everything in simulation. But as we all know, simulation and implementation can differ.}

\todo{something about testing the com unit}

\todo{8x8, 8x8x8 matrices}

\todo{something about testbench}

Testing uncovered many issues within the design

\subsection{Communication}

The communication unit was tested in hardware by connecting the output of the reception buffer to the input of the transmission buffer.
Then, sample data was sent over PCI Express to the SP605, before being then read back.

\subsection{Solved issues}

\emph{States and types were written to the wrong location.}
When writing single states or types, a half-row is read from BRAM, combined with the new data, and written back to BRAM.
Due to the usage of non-implementable code to specify how the data should be combined, the new data always ended up in the middle of the half-row.

\emph{Development ran indefinitely.}
Comparison of signals of different widths always return false.
Due to the comparison of a parameterized signal with a constant, the development unit would not iterate through the cells, and thus never finish.

\emph{WriteRule did not follow spesification.}
When the api transformed a 3D rule struct into an instruction, the position of up/down and north/south were swapped.

\emph{PrintTypes used wrong offsets for decomposition.}
When decomposing a row of types into individual types, an offset of 5 was used instead of 8.
This entailed that the printed values were completely wrong.

\emph{ReadVector and PrintVector used 32-bit words.}
The functions were not updated from using 32-bit to 64-bit words.
ReadVector would therefore expect twice the number of words provided by the hardware platform, causing the program to wait for nonexistant data.

\subsection{Remaining issues}

\emph{ReadStates prints garbage in top 32 bits.}
This occurs due to a buffering error in the LSS unit.
However, only the least 32 bits are used by the api, which means that this issue only impacts performance and not functionality.

\emph{WriteStates and WriteTypes does nothing in 3D.}
This issue is only present on the board, not in any simulations, which makes it hard to track down the cause.

\emph{ReadRuleVector sends incorrect data.}
The first execution of the instruction produces an extra word; a repetition of the first.
Following executions produces the correct amount of words, but the order is offset by one.

\emph{ReadUsedRules fails for SBM widths less than 16.}
The simulator crashes with an index-out-of-bounds error, due to the instruction treating a $[width\cdot4]$ bits wide signal as if it is 64 bits wide.
How ISE is able to implement the design despite illegal indexing is a mystery, but the instruction produces only zeroes when executed on the board.

\emph{Development rules does not activate.}
This issue is also only present on the board, making it hard to analyze.
The root cause could be with any of the following instructions: devstep, writeRule and setNumberOfLastRule.

\emph{Config does not properly write states in 3D.}
Simulations show that the BRAM address fluctuates \TODO.

\emph{Runstep causes every state to become zero.}
\TODO

\subsection{Current status}

\TODO

\begin{itemize}
    \item PCIe com unit works beautifully
    \item Large parts are non-functional
\end{itemize}

\begin{table}[!ht]
    \renewcommand{\arraystretch}{1.3}
    \caption{Implementational status of instructions}
    \label{tab:instructions}
    \centering
    \begin{tabular}{l|l|l}
        \bfseries Instruction & \bfseries Works in 2D & \bfseries Works in 3D \\
        \hline
        break & Yes & Yes \\
        clearBRAM & Yes & Yes \\
        config & Yes & Undecidable \\
        devstep & Undecidable & Undecidable \\
        doFitness & Undecidable & Undecidable \\
        end & Yes & Yes \\
        jump & Yes & Yes \\
        jumpEqual & Yes & Yes \\
        nop & Yes & Yes \\
        readback & Yes & Undecidable \\
        readFitness & Undecidable & Undecidable \\
        readRuleVector & No & No \\
        readState & Yes & Yes \\
        readStates & Yes & Yes \\
        readSums & Undecidable & Undecidable \\
        readType & Yes & Yes \\
        readTypes & Yes & Yes \\
        readUsedRules & Undecidable & No \\
        resetDevCounter & Yes & Yes \\
        run & Undecidable & Undecidable \\
        setNumberOfLastRule & Undecidable & Undecidable \\
        startDFT & Undecidable & Undecidable \\
        store & Yes & Yes \\
        switch & Yes & Yes \\
        writeLUTConv & Undecidable & Undecidable \\
        writeRule & Undecidable & Undecidable \\
        writeState & Yes & Yes \\
        writeStates & Yes & No \\
        writeType & Yes & Yes \\
        writeTypes & Yes & No \\
    \end{tabular}
\end{table}

