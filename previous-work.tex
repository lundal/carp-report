The Cellular Automata Research Platform has been the subject of three previous master theses at NTNU.
The original implementation was made by Djupdal in 2003.
It was then extended with a range of various output methods by Aamodt in 2005.
Finally, it was further extended and optimized in expectation of new hardware by Støvneng in 2014.

\subsection{Conception}

In 2002, NTNU invested in a CompactPCI computer with a NallaTech BenERA FPGA board to be used for research within the field of evolutionary hardware.
The task of developing a platform for the system, based on a matrix of sblocks, fell to Djupdal \cite{djupdal2003sblock}.

An overview of the resulting hardware platform is shown in \figurename~\ref{fig:overview-djupdal}.
It consists of the mentioned sblock matrix, BRAM for storing the state and type of each cell, a development unit, control logic, and a PCI communication unit.

\begin{figure}[!ht]
    \centering
    \includegraphics[width=0.48\textwidth]{figures/overview-djupdal}
    \caption{High-level block diagram of hardware platform after Djupdal's original work.}
    \label{fig:overview-djupdal}
\end{figure}

\todo{detailed description}

\todo{also software}

\subsection{Extension}

There was one major bottleneck in the original design.
To calculate the fitness of an individual, the state of each cell had to be transfered to the computer over the PCI interface.
Having a dedicated hardware unit would greatly improve the performance.
Additionaly, it was desired to have more information about the development process.
The task of realizing this fell to Aamodt \cite{aamodt2005sblock}.

An overview of the hardware platform with Aamodt additions is shown in \figurename~\ref{fig:overview-aamodt}.
The additions consists of a run-step function that calculates the number of live cells, BRAM to store the numbers, a fitness function, and two information outputs from the development unit.

\begin{figure}[!ht]
    \centering
    \includegraphics[width=0.48\textwidth]{figures/overview-aamodt}
    \caption{High-level block diagram of hardware platform after Aamodt's work. Additions are marked in green.}
    \label{fig:overview-aamodt}
\end{figure}

\todo{detailed description}

\todo{also software}

\subsection{Renovation: Støvneng \cite{stovneng2014sblock}}

\begin{itemize}
    \item New hardware
    \item Fig. \ref{fig:overview-stovneng}
\end{itemize}

\begin{figure}[!ht]
    \centering
    \includegraphics[width=0.48\textwidth]{figures/overview-stovneng}
    \caption{Støvneng's additions in green, optimizations and/or 3D modifications in orange}
    \label{fig:overview-stovneng}
\end{figure}

