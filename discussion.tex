\subsection{Challenges}

There was a lot of concern during initial hardware testing, as the SP605 was not detected by the computer.
Luckily, this proved not to be a hardware fault, but a mistake in the hardware setup guide; the position of SW1 was reversed.

The SP605 was pre-installed with an example design implementing communication over PCI Express with DMA.
However, the accompanying driver did not support newer linux kernels.
Additionaly, the design was written in verilog while CARP is written in vhdl, which meant extra effort to integrate the two.
There was some effort applied to update the driver, but it was abandoned due to near-untraceable segfaults.

The USB cable driver for usage of JTAG provided by Xilinx also had the problem of not being compatible with newer linux kernels.
Thankfully, a the driver found at \cite{usbdriver} is compatible and solves the problem.

\subsection{Current status}

\begin{itemize}
    \item PCIe com unit works
    \item Most of CA is functional
\end{itemize}

\subsection{Development bug}

\begin{itemize}
    \item Works in post-translate sim
    \item Doesn't work on dev board
    \item All post-map and post-par simulations give X'es
    \item Can't keep\_hierarchy to trace source
\end{itemize}

