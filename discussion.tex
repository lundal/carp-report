\subsection{Challenges}

There was a lot of concern during initial hardware testing, as the SP605 was not detected by the computer.
Luckily, this proved not to be a hardware fault, but a mistake in the hardware setup guide; the position of SW1 was reversed.

The SP605 was pre-installed with an example design implementing communication over PCI Express with DMA.
However, the accompanying driver did not support newer linux kernels.
Additionaly, the design was written in verilog while CARP is written in vhdl, which meant extra effort to integrate the two.
There was some effort applied to update the driver, but it was abandoned due to near-untraceable segfaults.

The USB cable driver for usage of JTAG provided by Xilinx also had the problem of not being compatible with newer linux kernels.
Thankfully, a the driver found at \cite{usbdriver} is compatible and solves the problem.

\todo{Development bug}

\begin{itemize}
    \item Works in post-translate sim
    \item Doesn't work on dev board
    \item All post-map and post-par simulations give X'es
    \item Can't keep\_hierarchy to trace source
\end{itemize}

\subsection{Future work}

The most important thing going forward is to fix the errors that are preventing the sblockmatrix and development units from working correctly.
However, this is no easy task, as large parts of the design are highly complex and difficult to debug.
The most extreme cases are the development and lss units which each consists of a single large file, around 1200 lines long, of complex pipelined code.
Simplification and modularization of these units is therefore imperative.

Another reason to simplify the development unit is it's extreme memory bandwidth requirement against the SBM BRAM.
Currently, the it is designed so that N rules are loaded, applied to every cell, then the N next rules loaded and so on.
This means that the SBM BRAM must supply 5 rows each cycle to test 1 row per cycle (or 8 for 2) in 3D, while N rules are needed per matrix iteration.
In addition, after the first pass, cells has to be read from BRAM B instead of BRAM A, since some cells might have changed.

A simplified process would be to read 5 rows, apply one rule to the center row each cycle, then read the next 5 rows, and so on.
This will greatly lessen the bandwidth requirements against the SBM BRAM, as the rows can be read in sequence while each rule is being applied.
Assuming there are more rules than the number of rows that must be read (highly likely), there is no performance loss.
Additionally, this would allow development to only read from BRAM A.

There are still some remains of having two clock domains, more specificaly a pair of flipflops used for clock-synchronization in the fetch and lss units.
It does not affect functionality, and has therefore been of low priority, but it does add a slight delay between the communication unit and the fetch and lss units when reading data.

Currently, the platform only supports dimension sizes that are powers of 2.
It would be beneficial to be able to select any size, allowing for fine-tuning of the resource usage, to get most out of the FPGA.

As noted in \cite{stovneng2014sblock}, Støvneng increased the base instruction and data sizes from 32 to 64 bits.
Although it is one way to accomodate for longer instructions, the decision is a little odd, considering both PCI and PCI Express are based on 32-bit word sizes.
This means that conversion is currently required between the LSS and communication units.
Since only 6 out of 30 instructions require more than 32 bits, communication could be simplified and optimized by going back to a 32-bit base size.

Finally, a unification of the 2D and 3D designs would be \TODO

\begin{itemize}
    \item Com buffer checks
    \item Get rid of tristate buffers \cite{koch2008buses}
    \item Get rid of reset (init values in \cite{ug687} page 55-56, \cite{wp272} page 5)
    \item Use generics instead of global constants
    \item Unify 2D and 3D designs
    \item Add instruction that reports info (2D/3D, SIZE, ++)
    \item Add general purpose counters
    \item DMA (better performance?)
    \item Toggle SBM wrap-around
\end{itemize}

With the current need for major fixes, simplification of development and lss, reducing the need for extreme memory bandwidth, removing tristates, removing resets, and unification, it might be a good idea to rebuild the platform from the ground up.
Starting from a clean slate, thoroughly evaluating every part of the design, replacing the bad features and improving the good, will likely result in a greatly improved platform for CA research.

