
\figurename~\ref{fig:overview-lundal} shows the changes to the hardware platform.
The old COM40 unit has been replaced by a new communication unit and a compatibility layer.
On the software side, the api uses linux' built-in drivers for PCI Express.

\begin{figure}[!ht]
    \centering
    \includegraphics[width=0.48\textwidth]{figures/overview-lundal}
    \caption{High-level block diagram of the current hardware platform. Additions are highlighted in green.}
    \label{fig:overview-lundal}
\end{figure}

\subsection{Detailed overview}

The new communication unit is based on Xilinx' reference PCI Express programmed io design.
It consists of the Xilinx PCI Express endpoint core, reception and transmission engines, data buffers, and a special request handler, as shown in \figurename~\ref{fig:details-communication}.

\begin{figure}[!ht]
    \centering
    \includegraphics[width=0.48\textwidth]{figures/details-communication}
    \caption{Detailed block-diagram of the PCI Express communication module.}
    \label{fig:details-communication}
\end{figure}

The endpoint core completely handles the physical and data link layers, and all TLPs related to configuration and establishment of the PCI Express connection.
Other TLPs, such as read and write requests, are presented on an AXI4-Stream interface \cite{ug672}.
The reception engine is responsible for parsing TLPs and either writing received data to the reception buffer or notifying the transmission engine about a read request.
The transmission engine is responsible for building completer TLPs to respond to read requests, using data from the transmission buffer.
The request handler listens to the read requests provided by the reception engine, and can override the transmission engine to respond to special requests.

\subsection{PCI Express Endpoint Core}

Several Spartan-6 FPGAs, including the one used in this project, contain a special-purpose hardware block for implementation of PCI Express.
The block completely handles the physical and data link layers, with the transaction layer left for the user.

To make use of the block, Xilinx provides the Spartan-6 Integrated PCI Express Endpoint Core; version 2.3 was used in this project.
This core additionally takes care of all TLPs related to configuration of the PCI Express connection.
Other TLPs, such as read and write requests, are presented on an AXI4-Stream interface \cite{ug672}.

The endpoint core is configured with two memory regions, both 4 kB in size\footnotemark.
\footnotetext{
    The smallest memory region that can be memory-mapped is one page. The default page size in Linux is 4 kB.
}
The first memory region (BAR0) is used for normal communication, while the second (BAR1) is used for special requests.
The separation is mostly conseptual as both regions are treated as one data stream.
The difference is that the special request handler kicks in for read requests to BAR1.

\subsection{Reception engine}

The reception engine is implemented as a simple state machine, as shown in \figurename~\ref{fig:statemachine-receive}.

\begin{figure}[!ht]
    \centering
    \includegraphics[width=0.48\textwidth]{figures/statemachine-receive}
    \caption{State machine for the reception engine.}
    \label{fig:statemachine-receive}
\end{figure}

Until the endpoint core presents valid data, the state machine remains in Idle.
When it does, the data is stored, and the TLP type is checked.
If it is a read or write request, the state machine continues down the corresponding path, otherwise the remaining data is discarded.
The remaining portion of the TLP headers are then parsed in the DW1 and DW2 states.
For read requests, the state machine waits in ReadWait until the transmission engine is ready to accept a new read request, and then proceeds to Idle.
For write requests, the state machine stays in WriteData, where one DW of data is written to the reception buffer each cycle, for the length of the packet, and then proceeds to Idle.

\subsection{Transmission engine}

The transmission engine is implemented as a simple state machine, as shown in \figurename~\ref{fig:statemachine-transmit}.

\begin{figure}[!ht]
    \centering
    \includegraphics[width=0.48\textwidth]{figures/statemachine-transmit}
    \caption{State machine for the transmission engine.}
    \label{fig:statemachine-transmit}
\end{figure}

Until the reception engine signals a read request, the state machine remains in Idle.
When a read request is signaled by the reception engine, the state machine begins to traverse the DW path.
The DW0, DW1 and DW2 states each transmit one DW of the completer TLP header.
Then if the special request signal is set, it procceds to CompleteSpecial, where it transmits data presented by the request handler.
Otherwise, it proceeds to CompleteData where it transmits one DW of data from the transmission buffer each cycle.
When the requested number of DWs has been transmitted it proceeds back to Idle.

\subsection{Request handler}

The request handler continually listens to the read requests presented by the reception engine.
If the request is targeting the primary memory area (BAR 0), it is a normal read request and the transmission engine is allowed to proceed as usual.
Otherwise, it is a special request and the transmission engine is overridden.

The kind of special request is determined by the address of the read request, and handled thereafter.
There are currently four special requests implemented, as shown in Table~\ref{tab:requests}.

\begin{table}[!ht]
    \renewcommand{\arraystretch}{1.3}
    \caption{Special requests}
    \label{tab:requests}
    \centering
    \begin{tabular}{c|l}
        \bfseries Address & \bfseries Request \\
        \hline
        0x00 & Get transmission buffer data count \\
        0x01 & Get transmission buffer available space \\
        0x02 & Get reception buffer data count \\
        0x03 & Get reception buffer available space \\
    \end{tabular}
\end{table}

Note that each of the implemented special requests assumes a read request length of one DW.
If the request has a greater length, the returned data is simply repeated to fill the packet.

\subsection{Buffers}

The buffers are implemented as first-in first-out (FIFO) queues using block RAM (BRAM) and two counters.
The counters determine the addresses that are written to and read from, and are incremented when the write or read signals are asserted.
\figurename~\ref{fig:wavediagram-fifo} shows how the FIFO is used to buffer two words.

\begin{figure}[!ht]
    \centering
    \includegraphics[width=0.40\textwidth]{figures/wavediagram-fifo}
    \caption{Wave diagram for the FIFO buffer, showing two consecutive writes followed by two consecutive reads.}
    \label{fig:wavediagram-fifo}
\end{figure}

Notice how the read signal needs to be asserted before the clock tick when data is read to ensure correct consecutive reads.
This is due to the BRAM used in the FIFO, which updates at clock ticks.
To have correct data available for a read in the following cycle, the address therefore has to be updated before the clock tick (by asserting the read signal).

\subsection{Compatibility layer}

\todo{rewrite this}

Due to the difference in wordsize between the old and new communication unit, a compatibility layer is added to allow other existing code to remain unchanged.

The compatibility layer is needed because the old COM40 unit was based on 64-bit data while PCI Express uses 32-bit data.



\subsection{Software API}

The communication part of the new software API is split into two parts.

The first is a general interface for connecting to PCI and PCI Express devices without using a custom driver.
It takes advantage of Linux' automatic population of /sys/devices/pci* with files representing the memory regions of all PCI and PCI Express devices.
The directory is searched by vendor and device id, and the corresponding memory regions is memory-mapped into the program.

The second is an interface specifically for the communication unit.
It provides open, close, read and write functions similar to the old BenERA interface, in addition to implementing all special request functions in Table~\ref{tab:requests}.
When a read or write operation is initiated, buffers are checked for available data or space.
If there is not enough present, the program waits and then rechecks.

