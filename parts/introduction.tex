\todo{in general about evolvable hardware}

The principles behind evolvable hardware and related technologies are presented in detail in Section~\ref{sec:background}.

Evolvable hardware has been an area of research at NTNU for more than a decade.
In 2002, NTNU invested in dedicated FPGA hardware for the usage as an evolutionary hardware platform \todo{fix sentence}.
The purpose was to create a platform for experimentation with evolution and development within a cellular automata (CA).

The initial work was done by Djupdal, before being extended by Aamodt.
The CA was implemented as a matrix of sblocks, reprogrammable CA cells designed specifically for usage with evolution.
It is connected to a development unit which can perform growth and change on the sblocks \todo{fix sentence}.
The hardware platform is connected to and controlled by a computer over a PCI connection.

A general use-case for the platform is to have the computer run a genetic algorithm, where the genotype represents the initial CA state and development rules.
Then, the CA is initialized and developed to produce a fenotype, which is executed to produce a fitness value.

In expectation of new hardware with a larger FPGA and faster PCI Express connection, Støvneng refurbished the design in 2014.
He took advantage of the added resources to greatly improve the performance of the platform, giving a speedup of 4 or more for most operations.
Additionally, he extended the CA into 3D and added a DFT.
However, since the hardware did not arrive in time, the new design was only tested in simulation and the communication interface was not upgraded.

An in-depth explanation of the platform's features, functionality and iterations is presented in Section~\ref{sec:previous-work}.

\todo{expand}

Section~\ref{sec:motivation} explains the motivations for continuing the work on the platform.

Section~\ref{sec:communication} goes into detail of the implementation of the communication unit.

Section~\ref{sec:verification} covers the verification method and results.

Section~\ref{sec:discussion} discusses the challenges tackled and future work.

Section~\ref{sec:conclusion} concludes this paper.

