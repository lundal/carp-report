Evolvable hardware (EHW) is a field of science where evolutionary algorithms (EAs) are used in the creation of specialized hardware.
EAs simulate mechanisms found in biological life, such as selection, reproduction and mutation.
The goal is to optimize fitness, the ability to survive in the competition for being the best solution.
EAs can quickly find approximate solutions to hard problems, and then gradually refine them into good solutions.

An interesting feature is that EAs can find ways to exploit hardware in ways that human designers cannot comprehend \cite{thompson1997evolved}.
This can be due to complex parallel interactions, or usage of properties that are not fully understood \cite{thompson1999analysis}.

Technologies related to EHW, including a common EA and hardware platforms are presented in Section~\ref{sec:background}.

Evolvable hardware has been an area of research at NTNU for more than a decade.
In 2002, NTNU invested in dedicated FPGA hardware with the intent of building an EHW platform.
The purpose was to create a platform for experimentation with evolution and development within a cellular automata (CA).

The initial work was done by Djupdal, before being extended by Aamodt.
The CA was implemented as a matrix of sblocks, reprogrammable CA cells designed specifically for usage with evolution.
It is connected to a development unit which can simulate growth and change in the sblock matrix.
The hardware platform is connected to and controlled by a computer over a PCI connection.

A general use-case for the platform is to have the computer run a genetic algorithm, where the genotype represents the initial CA state and development rules.
Then, the CA is initialized and developed to produce a phenotype, which is executed to produce a fitness value.

In expectation of new hardware with a larger FPGA and faster PCI Express connection, Støvneng refurbished the design in 2014.
He took advantage of the added resources to greatly improve the performance of the platform, giving a speedup of 4 or more for most operations.
Additionally, he extended the CA into 3D and added a DFT.
However, since the hardware did not arrive in time, the new design was only tested in simulation and the communication interface was not upgraded.

An in-depth explanation of the platform's features, functionality and iterations is presented in Section~\ref{sec:previous-work}.

The task of this project is to finish the new platform by implementing a new PCI Express communication unit, and to verify that everything is functional in hardware.
This will allow the new platform, which is both faster and more feature-rich, to be taken into use.
The motives are further detailed in Section~\ref{sec:motivation}.

The implementation of the new communication unit is detailed in Section~\ref{sec:communication}, while the verification process and results are detailed in Section~\ref{sec:verification}.

Challenges related to setting up the new hardware platform and testing is detailed in in Section~\ref{sec:discussion}, along with proposed design changes and other future work.

Section~\ref{sec:conclusion} concludes this paper.

