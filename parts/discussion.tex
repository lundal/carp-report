\todo{add section intro}

\subsection{Challenges}
\label{sec:challenges}

There was a lot of concern during initial hardware testing, as the SP605 was not detected by the computer.
A slightly curved circuit board led to the belief that there might be something wrong with the hardware.
Luckily, it proved not to be a hardware fault, but a mistake in the hardware setup guide; the position of SW1 was reversed, causing the board to operate in a completely different mode.

The SP605 was pre-installed with an example design implementing communication over PCI Express with DMA.
However, the accompanying driver did not support newer Linux kernels.
Additionaly, the design was written in verilog while CARP is written in VHDL, which meant extra effort to integrate the two.
There was some effort applied to update the driver, but it was abandoned due to near-untraceable segfaults.

The USB cable driver for usage of JTAG provided by Xilinx also had the problem of not being compatible with newer Linux kernels.
Thankfully, a third-party driver found at \cite{usbdriver} is compatible and solves the problem.

For unknown reasons, collisions occur on vital signals in all post-map and post-place-and-route simulations\footnotemark, causing them to be of no use.
\footnotetext{
    The order of the implementation process is: Synthesize, translate, map, place-and-route.
}
This makes it impossible to analyze issues that are present in implementation but not in post-translate simulation.
Since ISE will not respect the \emph{keep\_hierarchy}\footnotemark attribute for the unit in which the collision is first observed, tracing of the source has been unsuccessful.
\footnotetext{
    The \emph{keep\_hierarchy} attribute informs the synthesis tool that it should not flatten hardware design units to allow further optimizations.
    This is useful since the opimizations makes it near-impossible to trace signals.
}

\subsection{Future work}

The most important thing going forward is to fix the errors that are preventing the sblockmatrix and development units from working correctly.
However, this is no easy task, as large parts of the design are highly complex and difficult to debug.
The most extreme cases are the development and LSS units which each consists of a single large file, around 1200 lines long, of complex pipelined code.
Simplification and modularization of these units is therefore imperative.

Another reason to simplify the development unit is it's extreme memory bandwidth requirement against the SBM BRAM.
Currently, it is designed so that N rules are loaded, applied to every cell, then the N next rules loaded and so on.
This means that the BRAM must supply 5 rows each cycle to test 1 row per cycle (or 8 for 2) in 3D, while N rules are needed per matrix iteration.
In addition, after the first pass, center cells has to be read from BRAM B instead of BRAM A, since they might have changed.

A simplified process is to read 5 rows, apply one rule to the center row each cycle, then read the next 5 rows, and so on.
This will greatly lessen the bandwidth requirements against the BRAM, as new rows can be read in sequence while each rule is being applied to the current.
Assuming there are more rules than the number of rows that must be read (highly likely), there is no performance loss.
Additionally, this would allow development to only read from BRAM A, simplifying the dataflow.

There are still some remains of having two clock domains, more specificaly a pair of flipflops used for clock-synchronization in the fetch and lss units.
It does not affect functionality, and has therefore been of low priority, but it does add a slight delay between the communication unit and the fetch and lss units when reading data.

Currently, the platform only supports SBM sizes that are powers of 2.
It would be beneficial to be able to select any size, allowing for fine-tuning of the resource usage, to get most out of the FPGA.

As noted in \cite{stovneng2014sblock}, Støvneng increased the base instruction and data sizes from 32 to 64 bits.
Although it is one way to accomodate for longer instructions, the decision is a little odd, considering both PCI and PCI Express are based on 32-bit word sizes.
This means that conversion is currently required between the LSS and communication units.
Since only 6 out of 30 instructions require more than 32 bits, communication could be simplified and optimized by going back to a 32-bit base size.

The current design makes use of a lot of internal tristate buffers.
These are not supported in modern FPGAs \cite{koch2008buses}, and therefore need to be converted into other logic during synthesis.
The synthesis tool gently hints at this misuse by producing several warnings.
Removing tristate buffers will therefore result in code that more closely relates to its implementation.

Another feature that can beneficially be removed is the global asynchronous reset.
Since all Xilinx FPGAs start in a well defined state, a reset signal is only needed in very spesific cases \cite{ug687} \cite{wp272}.
Otherwise, it only serves to take up valuable resources.

Having the software interface automaticly adapt to the implemented hardware platform would be nifty.
This would remove the need to specify the sblock matrix size and type at compilation.
It could be accomplished by adding an instruction that returns the size of the sblock matrix.

A feature that could be interesting is the ability to enable or disable wrap-around for the edges of the sblock matrix.
Disabling it would mean that the size of the matrix can not be exploited to create an oscillating signal by something continually moving in one direction.

Finally, a unification of the 2D and 3D designs would give a more generic design and allow less code to be maintained.

With the current need for major fixes, simplification of development and LSS, reducing the need for extreme memory bandwidth, removing tristates, removing resets, and unification, it might be a good idea to rebuild the platform from the ground up.
Starting from a clean slate, thoroughly evaluating every part of the design, replacing the bad features and improving the good, will likely result in a greatly improved platform for CA research.

